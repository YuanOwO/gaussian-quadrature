\documentclass[a4paper, 10pt, oneside, fleqn]{article}

\usepackage{amsmath, amssymb, amsthm}       % 美國數學學會
\usepackage[CheckSingle, CJKmath]{xeCJK}    % 中文
\usepackage{fontspec}                       % 字型配置
\usepackage{geometry}                       % 版面配置
\usepackage[nocheck]{fancyhdr}              % 頁首頁尾
\usepackage{color}                          % 顏色
\usepackage[x11names]{xcolor}               % 更多顏色
\usepackage{datetime2}                      % 日期、時間
\usepackage{listings}                       % 顯示 code 用的
\usepackage{tikz}                           % 畫圖
\usepackage{enumerate}                      % 列舉環境
\usepackage{enumitem}                       % 自訂列舉環境
\usepackage{ulem}                           % 文字底線、刪除線
\usepackage{graphicx}                       % 插入圖片
\usepackage{multicol}                       % 多欄排版
\usepackage{hyperref}                       % 超連結
\usepackage{subcaption}                     % 子圖
\usepackage{cancel}                         % 取消線
\usepackage{multirow}                       % 表格合併列
\usepackage{float}                          % 浮動物件位置強制
\usepackage{notomath}                       % Noto 數學字體

%%%%%%%%%%%%%%%%%%%%%%%%%%%%%%

\setmainfont{Noto Serif}  % 主要字體

\setmonofont{Menlo}  % 等寬字體

\setCJKmainfont{Noto Serif TC}[  % 中文主要字體
    ItalicFont     = *-Regular, % 中文沒有斜體
    BoldItalicFont = *-Bold,    % 中文沒有斜體
]

\setCJKmonofont{Noto Sans TC}[  % 中文等寬字體
    ItalicFont     = *-Regular, % 中文沒有斜體
    BoldItalicFont = *-Bold,    % 中文沒有斜體
]

% \setCJKmonofont{LXGWWenKaiMonoTC}[  % 中文等寬字體
% Path           = .fonts/LXGW_WenKai_Mono_TC/,
% Extension      = .ttf,
% UprightFont    = *-Regular,
% BoldFont       = *-Bold,
% ]

%%%%%%%%%%%%%%%%%%%%%%%%%%%%%%

% 彩色模式
\newcommand{\keywordcolor}{\color{Blue1}}
\newcommand{\identifiercolor}{\color{black}}
\newcommand{\commentcolor}{\color{Red4}}
\newcommand{\stringcolor}{\color{Green4}}

% % 黑白模式
% \renewcommand{\keywordcolor}{\color{black}}
% \renewcommand{\identifiercolor}{\color{black}}
% \renewcommand{\commentcolor}{\color{black!70}}
% \renewcommand{\stringcolor}{\color{black!55}}

\makeatletter
\lst@CCPutMacro\lst@ProcessOther {"2D}{\lst@ttfamily{-{}}{-{}}}
\@empty\z@\@empty
\makeatother

\lstset{
    % language=C++,                         % Code 的語言
    numbers=none,                           % 行號的位置
    numberstyle=\footnotesize,              % 行號的字型和大小
    stepnumber=1,                           % 顯示行號的間隔
    numbersep=5pt,                          % 行號跟 code 的距離
    showspaces=false,                       % 顯示空白 (使用特別的底線記號)
    showtabs=false,                         % 顯示 Tab
    showstringspaces=false,                 % 顯示字串的空白
    tabsize=2,                              % Tab 的寬度
    frame=leftline,                         % adds a frame around the code
    captionpos=b,                           % sets the caption-position to bottom
    breaklines=true,                        % sets automatic line breaking
    breakatwhitespace=false,                % sets if automatic breaks should only happen at whitespace
    escapeinside={\%*}{*)},                 % if you want to add a comment within your code
    morekeywords={*},                       % 自訂的 Keywords
    basicstyle=\footnotesize\ttfamily,      % 基本的字型和大小
    backgroundcolor=\color{white},          % 背景顏色
    keywordstyle=\bfseries\keywordcolor,    % Keywords 的字型
    identifierstyle=\identifiercolor,       % 標示符的字型
    commentstyle=\itshape\commentcolor,     % 註解的字型
    stringstyle=\itshape\stringcolor,       % 字串的字型
}

\newcommand{\inputcpp}[ 1]{\lstinputlisting[language=C++]{#1}}      % 引入 C++
\newcommand{\inputpy}[ 1]{\lstinputlisting[language=Python]{#1}}    % 引入 Python
\newcommand{\inputjava}[ 1]{\lstinputlisting[language=Java]{#1}}    % 引入 Java
\newcommand{\inputtxt}[ 1]{\lstinputlisting{#1}}                    % 引入文字檔

%%%%%%%%%%%%%%%%%%%%%%%%%%%%%%

\geometry{a4paper}                  % 頁面 A4 大小
% \geometry{includehead}            % 內文區塊包含頁首,沒有頁尾、旁注
\geometry{headsep=5mm}              % 頁首和內文的距離
\geometry{vmargin=1in, hmargin=1in} % 垂直、水平邊界

\setlength{\parskip}{1.25ex}        % 段落間距

\XeTeXlinebreaklocale "zh"          % 中文自動換行
\XeTeXlinebreakskip = 0pt plus 1pt  % 設定段落之間的距離

\renewcommand{\arraystretch}{1.5}   % 表格列高
\setlength{\tabcolsep}{1.25ex}      % 表格欄寬

%%%%%%%%%%%%%%%%%%%%%%%%%%%%%%

\hypersetup{
    colorlinks=true,         % 啟用顏色連結
    linkcolor=red,           % 內部連結顏色
    filecolor=green,         % 檔案連結顏色
    urlcolor=blue,           % 網址顏色
    pdfborder={0 0 1}        % 設定邊框樣式(0 0 1 表示只有底部有邊框)
}

%%%%%%%%%%%%%%%%%%%%%%%%%%%%%%

\newtheorem{theorem}{Theorem}
\newtheorem{corollary}[theorem]{Corollary}
\newtheorem{lemma}[theorem]{Lemma}
\newtheorem{definition}[theorem]{Definition}
\newtheorem{conjecture}[theorem]{Conjecture}

\allowdisplaybreaks                     % 允許跨頁的多行公式

%%%%%%%%%%%%%%%%%%%%%%%%%%%%%%

\begin{document}
\pagestyle{fancy}

%%%%%%%%%%

% 頁首、頁尾
\renewcommand{\headrulewidth}{0.4pt}
\fancyhead[L]{Homework 3, 2D Gaussian Quadrature}
\fancyhead[R]{黃俊源 01257128}
\fancyfoot[C]{\thepage}

%%%%%%%%%%%%%%%%%%%%%%%%%%%%%%

\setlength{\abovedisplayskip}{8pt}
\setlength{\belowdisplayskip}{8pt}
\setlength{\abovedisplayshortskip}{4pt}
\setlength{\belowdisplayshortskip}{4pt}

\newcommand{\domain}{\boldsymbol{D}}
\newcommand{\jacobian}{\boldsymbol{J}}
\newcommand{\sci}[2]{#1 \times 10^{#2}}

%%%%%%%%%%%%%%%%%%%%%%%%%%%%%%

\section*{積分計算}

考慮以下二元函數 $f(x, y)$ 及積分區域 $\domain$:
\begin{align*}
    f(x, y) & = 3 \times \sin(8 \pi x) \times \cos(8 \pi y) + x + y + 1 \\
    \domain & = [2, 6] \times [2, 6]
\end{align*}

\begin{figure}[h]
    \centering
    \includegraphics[width=.75\linewidth]{image/contour_plot.png}
    \caption{函數 $f(x, y)$ 的等高線圖}
    \label{fig:contour_plot}
\end{figure}

\subsection*{解析解}

注意到 $\sin(8 \pi x)$ 與 $\cos(8 \pi y)$ 的週期皆為 $\frac{1}{4}$,
而積分區域 $\domain$ 在 $x$ 與 $y$ 方向上長度皆為 $4$,恰好包含了 $16$ 個完整的週期,
因此積分結果中關於三角函數的部分會相互抵銷,即
\begin{equation}
    \iint_{\domain} 3 \sin(8 \pi x) \cos(8 \pi y) \, dy \, dx = 0
\end{equation}

因此只需計算剩餘部分的積分:
\begin{align}
    \iint_{\domain} f(x, y) \, dy \, dx & =
    \iint_{\domain} \left[ 3 \sin(8 \pi x) \cos(8 \pi y) + x + y + 1 \right] dy \, dx
    \\
                                        & =
    \int_{2}^{6} \int_{2}^{6} (x + y + 1) \, dy \, dx
    \\
                                        & =
    \int_{2}^{6} \left[ (x + 1) y + \frac{y^2}{2} \right]_{2}^{6} dx
    \\
                                        & =
    \int_{2}^{6} \left[ (x + 1)(6 - 2) + \frac{36 - 4}{2} \right] dx
    \\
                                        & =
    \int_{2}^{6} \left[ 4x + 20 \right] dx
    \\
                                        & =
    \left[ 2x^2 + 20x \right]_{2}^{6}
    \\
                                        & =
    2 \times (36 - 4) + 20 \times (6 - 2) = 64 + 80 = 144
\end{align}

所以 $f(x, y)$ 在區域 $\domain$ 上積分的解析解為 $144$。

\subsection*{數值解:梯形法則}

將區域 $\domain$ 等分成 $n \times n$ 個小區域,則每個小區域的邊長為 $h = \frac{4}{n}$。
根據梯形法則,雙重積分的數值解可表示為
\begin{equation}
    \iint_{\domain} f(x, y) \, dy \, dx \approx
    \frac{h^2}{4} \sum_{i=0}^{n} \sum_{j=0}^{n} c_{ij} f(x_i, y_j)
\end{equation}
其中 $x_i = 2 + i \cdot h$,$y_j = 2 + j \cdot h$,且 $c_{ij}$ 為
\begin{equation}
    c_{ij} =
    \begin{cases}
        1, & \text{if } (i, j) \text{ 是四個角落的點}      \\
        2, & \text{if } (i, j) \text{ 是在邊界上但不是角落的點} \\
        4, & \text{if } (i, j) \text{ 是內部的點}
    \end{cases}
\end{equation}

\begin{lstlisting}[language=C++]
/**
 * 使用梯形法則計算 f(x, y) 在區域 [a, b] x [c, d] 上使用 n x m 分割的雙重積分值
 */
double trapezoid(double a, double b, double c, double d, int n, int m) {
    double hx = (b - a) / n, hy = (d - c) / m;
    double integral = 0.0;

    for (int i = 0; i <= n; i++) {
        for (int j = 0; j <= m; j++) {
            double x = a + i * hx, y = c + j * hy;
            double c = 4.0;  // 預設為內部點
            if (i == 0 || i == n) c /= 2.0;  // 在 x 邊界上
            if (j == 0 || j == m) c /= 2.0;  // 在 y 邊界上
            integral += c * f(x, y);
        }
    }

    return integral * (hx * hy / 4.0);
}
\end{lstlisting}

使用此方法計算當 $n = 4, 8, 16, 32, 64$ 時的數值解如下表(\autoref{tab:trapezoid})所示。


\begin{table}[h]
    \centering
    \begin{tabular}{cccc}
        \hline
        $n$ & 數值解                   & 絕對誤差                  & 相對誤差                  \\
        \hline
        4   & $143.999999999999812$ & $\sci{1.880995}{-13}$ & $\sci{1.306247}{-15}$ \\
        8   & $143.999999999999897$ & $\sci{1.028344}{-13}$ & $\sci{7.141278}{-16}$ \\
        16  & $143.999999999999961$ & $\sci{3.901046}{-14}$ & $\sci{2.709060}{-16}$ \\
        32  & $144.000000000000000$ & $\sci{1.387779}{-17}$ & $\sci{9.637353}{-20}$ \\
        64  & $144.000000000000000$ & $\sci{2.775558}{-17}$ & $\sci{1.927471}{-19}$ \\
        \hline
    \end{tabular}
    \caption{梯形法則積分結果與誤差比較(已知解析解 = $144$)}
    \label{tab:trapezoid}
\end{table}

\subsection*{Gaussian Quadrature}

首先,我們將積分區域 $\domain$ 切分成若干個小區域(cell),並對每個 cell 使用 $n \times n$ 個 Gaussian-Legendre 節點和其對應的權重來進行積分。最後,將所有 cells 的積分結果相加,即可得到整個區域的積分結果。

假設某一 cell 為 $[a, b] \times [c, d]$,則需要將 Gaussian 節點從標準區間 $[-1, 1]$ 映射到該 cell 上。映射公式如下所示:
\begin{align}
    x(\xi)  & = \frac{b - a}{2} \xi + \frac{a + b}{2}  \\
    y(\eta) & = \frac{d - c}{2} \eta + \frac{c + d}{2}
\end{align}
其中 $\xi, \eta$ 為標準區間 $[-1, 1]$ 上的 Gaussian 節點,並且有 Jacobian 行列式
\begin{equation}
    \lvert \jacobian \rvert = \frac{(b - a)(d - c)}{4}
\end{equation}

因此,該 cell 上的積分可以表示為
\begin{equation}
    \iint_{\text{cell}} f(x, y) \, dy \, dx \approx
    \sum_{i=1}^{n} \sum_{j=1}^{n} w_i w_j f(x(\xi_i), y(\eta_j)) \lvert \jacobian \rvert
\end{equation}
其中 $\xi_i$ 與 $\eta_j$ 為 Legendre 多項式 $P_n(x)$ 的根,$w_i$ 與 $w_j$ 為對應的權重。

\begin{lstlisting}[language=C++]
/**
 * 使用 Gauss-Legendre 積分公式計算 f(x, y) 在區域 [a, b] x [c, d] 上的雙重積分值
 */
long double integrate_cell_gauss_legendre(long double a, long double b,
                                          long double c, long double d,
                                          const std::vector<long double>& roots,
                                          const std::vector<long double>& weights) {
    long double integral = 0.0L;

    for (int i = 0; i < n; i++) {
        for (int j = 0; j < n; j++) {
            // 將 Gaussian 節點轉換為區域 [a,b] x [c,d] 上的座標
            long double x = ((b - a) / 2.0L) * roots[i] + (a + b) / 2.0L;
            long double y = ((d - c) / 2.0L) * roots[j] + (c + d) / 2.0L;

            // 累加對應的權重和函數值
            integral += weights[i] * weights[j] * f(x, y);
        }
    }

    // 乘上 Jacobian 行列式
    return integral * ((b - a) / 2.0L) * ((d - c) / 2.0L);
}
\end{lstlisting}

最後,將所有 cell 的積分結果相加,即可得到整個區域 $\domain$ 的積分結果
\begin{equation}
    \iint_{\domain} f(x, y) \, dy \, dx \approx
    \sum_{\text{cells}} \iint_{\text{cell}} f(x, y) \, dy \, dx
\end{equation}

\begin{lstlisting}[language=C++]
/**
 * 使用 Gauss-Legendre 積分公式計算 f(x, y) 在區域 [a, b] x [c, d] 上
 * 使用 n 個節點和 meshs x meshs 分割的雙重積分值
 */
long double integrate_gauss_legendre(long double a, long double b,
                                     long double c, long double d,
                                     int n, int meshs) {
    // 取得 Legendre 多項式 P_n(x) 的根與權重
    auto [roots, weights] = legendre_roots_weights(n);
    // 計算每個網格單元的大小
    long double hx = (b - a) / meshs, hy = (d - c) / meshs;
    long double integral = 0.0L;

    // 對每個 cell 進行積分並累加結果
    for (int i = 0; i < meshs; i++) {
        for (int j = 0; j < meshs; j++) {
            // 計算每個 cell 的邊界
            long double cell_a = a + i * hx, cell_b = cell_a + hx;
            long double cell_c = c + j * hy, cell_d = cell_c + hy;

            // 計算該 cell 的積分並累加
            integral += integrate_cell_gauss_legendre(cell_a, cell_b, cell_c, cell_d, roots, weights);
        }
    }

    return integral;
}
\end{lstlisting}

在主程式中,我們可以設定多組不同的區塊數量 $M$ 和 Gaussian 節點數量 $N$(亦即 Legendre 多項式的次數),並使用 Gauss-Legendre 積分公式來計算每一組參數下的積分結果。計算完成後,將其與解析解進行比較,並計算相應的誤差。此外,在計算過程中,還會記錄執行 100 次所花費的時間,以便評估不同參數組合下的效能表現。

\begin{lstlisting}[language=C++]
std::vector<int> mesh_values = {1, 2, 4, 8, 16, 32, 64};  // 要測試的 mesh
std::vector<int> n_values = {1, 2, 3, 4, 5, 6, 7};        // 要測試的 n 值

for (auto&& m : mesh_values) {
    for (auto&& n : n_values) {
        long double result;
        auto start = std::chrono::high_resolution_clock::now(); // 開始計時

        for (int iter = 0; iter < 100; iter++)
            result = integrate_gauss_legendre(2.0L, 6.0L, 2.0L, 6.0L, n, m);

        auto end = std::chrono::high_resolution_clock::now();   // 結束計時
        auto elapsed = end - start;

        using time_unit = std::chrono::duration<long double, std::micro>;  // 微秒
        long double time = std::chrono::duration_cast<time_unit>(elapsed).count() / 100;

        long double error = result - 144.0L;  // 已知解析解為 144

        std::cout << "mesh = " << m << ", n = " << n << ", result = " << result
                  << ", err = " << error << ", rel_err = " << error / 144.0L
                  << ", time = " << time << " µs" << std::endl;
    }
    std::cout << "----------------------------------------" << std::endl;
}
\end{lstlisting}

\section*{成果}

程式對於每組 $M$ 和 $N$ 計算出數值積分結果後,將其與解析解 $144$ 進行比較,並計算出絕對誤差與相對誤差,同時記錄每組參數下的平均執行時間。執行結果請見下頁表格 \autoref{tab:gauss_legendre},從表格中我們可以觀察到不同網格數量與 Gaussian 節點數量對積分結果精度與計算時間的影響,如下所示:

\begin{enumerate}
    \item 隨著區塊數量 $M$ 的增加,積分值逐漸接近解析解 $144$,絕對誤差也逐漸減小,最終相對誤差接近 $10^{-19}$ \~{} $10^{-18}$ 的量級,相當於 \verb|long double| 的機器精度限制。然而,當 $M \geq 16$ 時,誤差的變化趨於平緩,變化幅度變得不明顯。
    \item 隨著區塊數量的增加,計算所需的執行時間顯著增長,從微秒級別增長到毫秒級別。
    \item 在較小的區塊數量下,增加 Gaussian 節點數量 $N$ 可以顯著提高積分的精度,減少誤差。然而,當區塊數量達到一定程度(例如 $M \geq 16$)後,進一步增加 $N$ 對誤差的改善效果變得不顯著。
    \item 增加節點數量 $N$ 會導致執行時間增加,因為更多的節點意味著更多的計算步驟。
\end{enumerate}

\begin{table}[H]
    \centering
    \begin{tabular}{cccccc}
        \hline
        Mesh                & N & 高斯積分值                 & 絕對誤差                 & 相對誤差                 & 執行時間     \\
        \hline
        \multirow{5}{*}{1}  & 1 & $143.999999999999812$ & $\sci{1.881}{-13}$   & $\sci{1.30625}{-15}$ & 1.241 µs \\
                            & 2 & $143.999999999999897$ & $\sci{1.03029}{-13}$ & $\sci{7.15477}{-16}$ & 1.685 µs \\
                            & 3 & $143.999999999999927$ & $\sci{7.34829}{-14}$ & $\sci{5.10298}{-16}$ & 3.321 µs \\
                            & 4 & $143.999999999999990$ & $\sci{1.02557}{-14}$ & $\sci{7.122}{-17}$   & 3.853 µs \\
                            & 5 & $143.999999999999991$ & $\sci{8.89566}{-15}$ & $\sci{6.17754}{-17}$ & 11.47 µs \\
        \hline
        \multirow{5}{*}{2}  & 1 & $143.999999999999812$ & $\sci{1.88113}{-13}$ & $\sci{1.30634}{-15}$ & 1.31 µs  \\
                            & 2 & $143.999999999999975$ & $\sci{2.53131}{-14}$ & $\sci{1.75785}{-16}$ & 2.152 µs \\
                            & 3 & $143.999999999999917$ & $\sci{8.29892}{-14}$ & $\sci{5.76314}{-16}$ & 4.23 µs  \\
                            & 4 & $143.999999999999938$ & $\sci{6.20476}{-14}$ & $\sci{4.30886}{-16}$ & 5.941 µs \\
                            & 5 & $143.999999999999978$ & $\sci{2.15938}{-14}$ & $\sci{1.49957}{-16}$ & 13.23 µs \\
        \hline
        \multirow{5}{*}{4}  & 1 & $143.999999999999982$ & $\sci{1.75832}{-14}$ & $\sci{1.22105}{-16}$ & 1.991 µs \\
                            & 2 & $143.999999999999941$ & $\sci{5.86198}{-14}$ & $\sci{4.07082}{-16}$ & 3.882 µs \\
                            & 3 & $144.000000000000002$ & $\sci{1.79023}{-15}$ & $\sci{1.24322}{-17}$ & 8.123 µs \\
                            & 4 & $144.000000000000001$ & $\sci{6.80012}{-16}$ & $\sci{4.7223}{-18}$  & 12.18 µs \\
                            & 5 & $144.000000000000012$ & $\sci{1.22263}{-14}$ & $\sci{8.49051}{-17}$ & 23.05 µs \\
        \hline
        \multirow{5}{*}{8}  & 1 & $144.000000000000025$ & $\sci{2.5091}{-14}$  & $\sci{1.74243}{-16}$ & 11.06 µs \\
                            & 2 & $143.999999999999919$ & $\sci{8.06855}{-14}$ & $\sci{5.60316}{-16}$ & 12.38 µs \\
                            & 3 & $144.000000000000006$ & $\sci{6.17562}{-15}$ & $\sci{4.28862}{-17}$ & 24.27 µs \\
                            & 4 & $144.000000000000000$ & $\sci{2.77556}{-16}$ & $\sci{1.92747}{-18}$ & 58.72 µs \\
                            & 5 & $144.000000000000000$ & $\sci{4.71845}{-16}$ & $\sci{3.2767}{-18}$  & 64.84 µs \\
        \hline
        \multirow{5}{*}{16} & 1 & $144.000000000000046$ & $\sci{4.64351}{-14}$ & $\sci{3.22466}{-16}$ & 14.19 µs \\
                            & 2 & $143.999999999999998$ & $\sci{2.1233}{-15}$  & $\sci{1.47451}{-17}$ & 39.54 µs \\
                            & 3 & $144.000000000000001$ & $\sci{6.80012}{-16}$ & $\sci{4.7223}{-18}$  & 82.74 µs \\
                            & 4 & $144.000000000000000$ & $\sci{1.38778}{-17}$ & $\sci{9.63735}{-20}$ & 146.7 µs \\
                            & 5 & $144.000000000000000$ & $\sci{1.38778}{-17}$ & $\sci{9.63735}{-20}$ & 220.3 µs \\
        \hline
        \multirow{5}{*}{32} & 1 & $144.000000000000000$ & $\sci{2.77556}{-17}$ & $\sci{1.92747}{-19}$ & 53.9 µs  \\
                            & 2 & $144.000000000000000$ & $\sci{1.38778}{-17}$ & $\sci{9.63735}{-20}$ & 151.9 µs \\
                            & 3 & $144.000000000000000$ & $\sci{2.77556}{-17}$ & $\sci{1.92747}{-19}$ & 332.1 µs \\
                            & 4 & $144.000000000000000$ & $\sci{2.77556}{-17}$ & $\sci{1.92747}{-19}$ & 579.5 µs \\
                            & 5 & $144.000000000000000$ & $\sci{1.38778}{-17}$ & $\sci{9.63735}{-20}$ & 859.9 µs \\
        \hline
        \multirow{5}{*}{64} & 1 & $144.000000000000000$ & $\sci{2.77556}{-17}$ & $\sci{1.92747}{-19}$ & 207.8 µs \\
                            & 2 & $144.000000000000000$ & $\sci{1.249}{-16}$   & $\sci{8.67362}{-19}$ & 603.8 µs \\
                            & 3 & $144.000000000000000$ & $\sci{8.32667}{-17}$ & $\sci{5.78241}{-19}$ & 1251 µs  \\
                            & 4 & $144.000000000000000$ & $\sci{1.249}{-16}$   & $\sci{8.67362}{-19}$ & 2178 µs  \\
                            & 5 & $144.000000000000000$ & $\sci{8.32667}{-17}$ & $\sci{5.78241}{-19}$ & 3332 µs  \\
        \hline
    \end{tabular}
    \caption{Gauss-Legendre 積分結果與誤差比較(已知解析解 = $144$)}
    \label{tab:gauss_legendre}
\end{table}

\subsection*{H-refinement}

\begin{figure}[h]
    \centering
    \includegraphics[width=.75\linewidth]{image/h_refinement.png}
    \caption{分割空間成更多小區塊對相對誤差的影響}
    \label{fig:h_refinement}
\end{figure}

從上圖 \autoref{fig:h_refinement} 可見,在區塊數量 $M$ 較小時(如 $M \leq 16$),相對誤差的波動非常劇烈,這說明在只有少量區塊的情況下,積分結果對於區塊劃分的敏感度較高,可能會導致較大的誤差。

隨著 $M$ 的增加,相對誤差顯著下降並趨於穩定,主要集中在 $10^{-19}$ 到 $10^{-18}$ 的範圍內,這表明增加區塊數量能有效提高幾分的穩定性和精度。特別是在 $M \geq 24$ 之後,使用不同取樣點數量 $N = 2, 3, 4$ 的三種方法,其相對誤差幾乎趨近一致,這說明了 h-refinement 已經將誤差逼近到 \verb|long double| 的機器精度極限了。在到達此階段後,繼續增加 $M$ 對誤差的改善效果將不再顯著。

\subsection*{P-refinement}

\begin{figure}[h]
    \centering
    \includegraphics[width=.75\linewidth]{image/p_refinement.png}
    \caption{固定 mesh,增加取樣點數量相對誤差的影響}
    \label{fig:p_refinement}
\end{figure}

從上圖 \autoref{fig:p_refinement} 可見與 h-refinement 類似的,當取樣點數量 $N$ 較少時(例如 $N \leq 3$),相對誤差的波動較大,這說明在取樣點數量不足的情況下,積分結果對於取樣點的選擇較為敏感。

當 $N$ 增加到一定程度後(如 $N \geq 16$),在低區塊數量 $M = 16, 32$ 的情況下,相對誤差趨近於穩定在 $10^{-18}$ 到 $10^{-17}$ 左右。雖然將區塊數量從 $M = 32$ 提高到 $64$ 時可以使相對誤差降低大約一個量級,但是這項改進是以執行時間大幅增加為代價的。這表示 p-refinement 在達到一定程度後,進一步增加 $N$ 對誤差的改善效果不再顯著,而且會導致計算成本急劇攀升。

\subsection*{比較 h-refinement 與 p-refinement}

在 Gaussian Quadrature 中,h-refinement 與 p-refinement 是兩種提升計算精度的關鍵策略,兩者各有優勢並在精度與效率之間扮演不同的角色。

H-refinement 透過將積分區域切割成更多、更細的區塊($M$)來運作。這種方式能有效降低每個 cell 的局部誤差,並最終使整體精度逼近機器精度的下限。H-refinement 的優勢在於能提升最終可達到的最高準確度,但其收斂速度屬於代數級,且區塊數量 $M$ 增加後,計算量亦會大幅成長,因此計算成本較高。

相對地,p-refinement 則是在每個 cell 內增加高斯取樣點($N$),從而充分發揮高斯積分的指數級收斂性。這意味著只需少量增加取樣點 $N$,誤差便能迅速下降,使計算成本相對更具效率。然而,p-refinement 的精度極限會受到區塊解析度 $M$ 的限制;當 $M$ 不足時,p-refinement 無法突破由區塊數量下限所設定的誤差瓶頸。

兩者之間的依賴關係呈現出明確的分工:H-refinement 提供了最終準確度的上限,決定了誤差能否逼近機器精度;而 p-refinement 提供了快速收斂的能力,決定了在有限計算時間內能否有效降低誤差。

整體而言,在本題這種針對平滑函數進行積分的情況下,p-refinement 的高效率與快速收斂特性更具實際價值。因此,在大多數需要兼顧準確度與計算成本的情境中,p-refinement 的重要性略高於 h-refinement。然而,在追求極限精度的應用中,兩者仍需共同作用,才能達到最佳的數值表現。

\section*{計算 Legendre 多項式根與權重}

在這次作業中,我參考了 \href{https://github.com/numpy/numpy}{Numpy} 的相關原始碼實作,使用 Golub-Welsch 演算法來計算 Legendre 多項式 $P_n(x)$ 的根與權重。

在這裡我們使用二維 \verb|vector| 來表示矩陣:
\begin{lstlisting}[language=C++]
using Matrix = std::vector<std::vector<long double>>;
\end{lstlisting}

首先,我們需要建立一個大小為 $n \times n$ 的對稱三對角矩陣 $\jacobian$,其為多項式 $P_n(x)$ 的伴隨矩陣。對於 Legendre 多項式,矩陣 $\jacobian$ 的對角線元素皆為 $0$,而次對角線元素則為 $\beta_k = \frac{k}{\sqrt{4k^2 - 1}}$,其中 $k = 1, 2, \ldots, n-1$。矩陣 $\jacobian$ 的形式如下所示:
\begin{equation}
    \jacobian =
    \begin{bmatrix}
        0       & \beta_1 & 0       & \cdots & 0           & 0           \\
        \beta_1 & 0       & \beta_2 & \cdots & 0           & 0           \\
        0       & \beta_2 & 0       & \cdots & 0           & 0           \\
        \vdots  & \vdots  & \vdots  & \ddots & \vdots      & \vdots      \\
        0       & 0       & 0       & \cdots & 0           & \beta_{k-1} \\
        0       & 0       & 0       & \cdots & \beta_{k-1} & 0
    \end{bmatrix}_{n \times n}
\end{equation}

\begin{lstlisting}[language=C++]
/**
 * 產生 n 次 Legendre 多項式的伴隨矩陣
 */
Matrix legendre_companion_matrix(int n) {
    Matrix J(n, std::vector<long double>(n, 0.0L));

    // 填充 beta_k = k / sqrt(4k^2 - 1)
    for (int k = 1; k < n; k++) {
        long double beta = k / std::sqrt(4.0L * k * k - 1.0L);
        J[k][k - 1] = J[k - 1][k] = beta;
    }

    return J;
}
\end{lstlisting}

恰好矩陣 $\jacobian$ 的特徵值即為 Legendre 多項式 $P_n(x)$ 的根,而對應的特徵向量則可用來計算權重。接著,我們可以使用線性代數的方法(如 QR 分解、Jacobi 方法等)來求解矩陣 $\jacobian$ 的特徵值,即可獲得所需的所有根。其中我設定最大迭代次數為 $1000n$,並以長雙精度浮點數的機器精度作為收斂閾值。

\begin{lstlisting}[language=C++]
/**
 * 使用 Jacobi 方法計算對稱矩陣的特徵值
 */
std::vector<long double> jacobi_eigen(const Matrix& _A) {
    const long double eps = LDBL_EPSILON;    // 收斂閾值
    const int max_iter = 1000 * n;  // 最大迭代次數
    // 這裡應該實作特徵值分解的數值方法
    // 返回特徵值陣列
}
\end{lstlisting}

由於我們使用 Jacobi 方法計算出初步的根後,這些根可能還不是非常精確,因此我們接下來使用牛頓法來修正,進一步提升根的精度。對於每個初始根 $x_i$,我們可以使用以下迭代公式進行更新:
\begin{equation}
    {x_i} \leftarrow x_i - \frac{P_n(x_i)}{P_n'(x_i)}
\end{equation}

其中 $P_n(x)$ 為 Legendre 多項式,可以透過遞迴關係計算:
\begin{align}
    P_0(x)     & = 1, \quad P_1(x) = x,                                                       \\
    P_{k+1}(x) & = \frac{(2k + 1)x P_k(x) - k P_{k-1}(x)}{k + 1}, \quad k = 1, 2, \ldots, n-1
\end{align}
導數 $P_n'(x)$ 也可以透過以下關係計算:
\begin{equation}
    P_n'(x) = \frac{n}{x^2 - 1} \left[ x P_n(x) - P_{n-1}(x) \right]
\end{equation}

\begin{lstlisting}[language=C++]
/**
 * 計算 Legendre 多項式 P_n(x)
 */
long double legendre_value(int n, long double x) {
    if (n == 0) return 1.0L;  // P_0(x) = 1
    if (n == 1) return x;     // P_1(x) = x
    long double Pn_2 = 1.0L, Pn_1 = x, Pn;  // P_0(x), P_1(x)

    // 使用遞迴公式從 P_2(x) 計算到 P_n(x)
    for (int k = 2; k <= n; k++) {
        Pn = ((2.0L * k - 1.0L) * x * Pn_1 - (k - 1.0L) * Pn_2) / k;
        Pn_2 = Pn_1; Pn_1 = Pn;
    }
    return Pn;
}

/**
 * 計算 Legendre 多項式 P_n(x) 的導數 P_n'(x)
 */
long double legendre_derivative(int n, long double x) {
    if (n == 0) return 0.0L;  // P_0(x) = 1 -> P_0'(x) = 0
    if (n == 1) return 1.0L;  // P_1(x) = x -> P_1'(x) = 1
    long double Pn = legendre_value(n, x);
    long double Pn_1 = legendre_value(n - 1, x);
    return n / (x * x - 1.0L) * (x * Pn - Pn_1);
}
\end{lstlisting}

得到修正後的根後,我們可以計算對應的權重 $w_i$,其計算公式如下:
\begin{equation}
    w_i = \frac{2}{(1 - x_i^2) [P_n'(x_i)]^2}
\end{equation}

由於 Legendre 多項式是偶對稱的,我們知道它的根是對稱的,即對於每個根 $x_i$,都有一個對應的根 $-x_i$。因此,在數值積分中,我們可以對計算出的根和權重進行對稱化:

\begin{itemize}
    \item 對稱化根:對於每一對根 $(-x_i, x_i)$,我們可以將它們合併成一個對稱的表示。
    \item 對稱化權重:對於每一對根對應的權重,我們可以將它們進行平均,這樣保證積分結果對稱。
\end{itemize}

最後,為了確保 Gaussian Quadrature的結果正確,我們需要對權重進行縮放。由於 Gaussian Quadrature 要求權重的總和等於 2,我們需要對權重進行正規化處理:

\begin{equation}
    w_i \leftarrow \frac{w_i}{\sum w_i} \times 2
\end{equation}

範例程式碼如下所示:
\begin{lstlisting}[language=C++]
/**
 * 計算 n 次 Legendre 多項式的根和權重
 */
std::pair<std::vector<long double>, std::vector<long double>> legendre_roots_weights(int n) {
    // first approximation of roots. We use the fact that the companion
    // matrix is symmetric in this case in order to obtain better zeros.
    Matrix J = legendre_companion_matrix(n);
    std::vector<long double> roots = jacobi_eigen(J);
    std::sort(roots.begin(), roots.end());  // 將 roots 由小排到大

    std::vector<long double> dy(n), df(n), fm(n), weights(n);
    long double max_df = 0, max_fm = 0;

    // improve roots by one application of Newton's method
    for (int i = 0; i < n; i++) {
        dy[i] = legendre_value(n, roots[i]);
        df[i] = legendre_derivative(n, roots[i]);
        roots[i] -= dy[i] / df[i];
        max_df = std::max(max_df, std::fabs(df[i]));
    }

    // compute the weights. We scale the factor to avoid possible numerical
    // overflow.
    for (int i = 0; i < n; i++) {
        fm[i] = legendre_value(n - 1, roots[i]);
        max_fm = std::max(max_fm, std::fabs(fm[i]));
    }

    for (int i = 0; i < n; i++)
        fm[i] /= max_fm, df[i] /= max_df;
    for (int i = 0; i < n; i++)
        weights[i] = 1.0L / (fm[i] * df[i]);

    // for Legendre we can also symmetrize the roots and weights
    std::vector<long double> w_copy = weights, r_copy = roots;
    long double w_sum = 0.0L;
    for (int i = 0; i < n; i++) {
        weights[i] = (w_copy[i] + w_copy[n - 1 - i]) / 2.0L;
        roots[i] = (r_copy[i] - r_copy[n - 1 - i]) / 2.0L;
        w_sum += weights[i];
    }

    // scale weights to get the right value
    for (int i = 0; i < n; i++)
        weights[i] *= 2.0L / w_sum;

    return {roots, weights};
}
\end{lstlisting}


\end{document}
